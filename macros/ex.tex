% Exemples
% Macros relatives à la traduction de PH avec arcs neutralisants vers PH à k-priorités fixes

% Macros générales
%\newcommand{\ie}{\textit{i.e.} }
\newcommand{\segm}[2]{\llbracket #1; #2 \rrbracket}
%\newcommand{\f}[1]{\mathsf{#1}}

% Notations générales pour PH
\newcommand{\PH}{\mathcal{PH}}
%\newcommand{\PHs}{\mathcal{S}}
\newcommand{\PHs}{\Sigma}
%\newcommand{\PHp}{\mathcal{P}}
\newcommand{\PHp}{\textcolor{red}{\mathcal{P}}}
%\newcommand{\PHproc}{\mathcal{P}}
\newcommand{\PHproc}{\mathbf{Proc}}
\newcommand{\Proc}{\PHproc}
\newcommand{\PHh}{\mathcal{H}}
\newcommand{\PHa}{\PHh}
%\newcommand{\PHa}{\mathcal{A}}
\newcommand{\PHl}{\mathcal{L}}
\newcommand{\PHn}{\mathcal{N}}

\newcommand{\PHhitter}{\mathsf{hitter}}
\newcommand{\PHtarget}{\mathsf{target}}
\newcommand{\PHbounce}{\mathsf{bounce}}
%\newcommand{\PHsort}{\Sigma}
\newcommand{\PHsort}{\PHs}

%\newcommand{\PHfrappeur}{\mathsf{frappeur}}
%\newcommand{\PHcible}{\mathsf{cible}}
%\newcommand{\PHbond}{\mathsf{bond}}
%\newcommand{\PHsorte}{\mathsf{sorte}}
%\newcommand{\PHbloquant}{\mathsf{bloquante}}
%\newcommand{\PHbloque}{\mathsf{bloquee}}

%\newcommand{\PHfrappeR}{\textcolor{red}{\rightarrow}}
%\newcommand{\PHmonte}{\textcolor{red}{\Rsh}}

\newcommand{\PHhitA}{\rightarrow}
\newcommand{\PHhitB}{\Rsh}
%\newcommand{\PHfrappe}[3]{\mbox{$#1\PHhitA#2\PHhitB#3$}}
%\newcommand{\PHfrappebond}[2]{\mbox{$#1\PHhitB#2$}}
\newcommand{\PHhit}[3]{#1\PHhitA#2\PHhitB#3}
\newcommand{\PHfrappe}{\PHhit}
\newcommand{\PHhbounce}[2]{#1\PHhitB#2}
\newcommand{\PHobj}[2]{\mbox{$#1\PHhitB^*\!#2$}}
\newcommand{\PHobjectif}{\PHobj}
\newcommand{\PHconcat}{::}
%\newcommand{\PHneutralise}{\rtimes}
\def\Sce{\mathbf{Sce}}

% Actions plurielles
\newcommand{\PHhitmultsymbol}{\rightarrowtail}
\newcommand{\PHhitmult}[2]{\mbox{$#1 \PHhitmultsymbol #2$}}
\newcommand{\PHfrappemult}{\PHhitmult}
\newcommand{\PHfrappemults}[2]{\PHhitmult{\{#1\}}{\{#2\}}}

\def\PHget#1#2{{#1[#2]}}
%\newcommand{\PHchange}[2]{#1\langle #2 \rangle}
%\newcommand{\PHchange}[2]{(#1 \Lleftarrow #2)}
%\newcommand{\PHarcn}[2]{\mbox{$#1\PHneutralise#2$}}
\newcommand{\PHplay}{\cdot}

\newcommand{\PHstate}[1]{\mbox{$\langle #1 \rangle$}}
\newcommand{\PHetat}{\PHstate}

\def\supp{\mathsf{support}}
\def\first{\mathsf{first}}
\def\last{\mathsf{last}}

\def\DNtrans{\rightarrow_{ADN}}
\def\DNdef{(\mathbb F, \langle f^1, \dots, f^n\rangle)}
\def\DNdep{\mathsf{dep}}
\def\PHPtrans{\rightarrow_{PH}}
\def\get#1#2{#1[{#2}]}
\def\encodeF#1{\mathbf{#1}}
\def\toPH{\encodeF{PH}}
\def\card#1{|#1|}
\def\decode#1{\llbracket#1\rrbracket}
\def\encode#1{\llparenthesis#1\rrparenthesis}
\def\Hits{\PHa}
\def\hit{\PHhit}
\def\play{\cdot}

\def\Pint{\textsc{PINT}}



\usepackage{ifthen}

\newcommand{\currentScope}{}
\newcommand{\currentSort}{}
\newcommand{\currentSortLabel}{}
\newcommand{\currentAlign}{}
\newcommand{\currentSize}{}

\newcounter{la}
\newcommand{\TSetSortLabel}[2]{
  \expandafter\repcommand\expandafter{\csname TUserSort@#1\endcsname}{#2}
}
\newcommand{\TSort}[4]{
  \renewcommand{\currentScope}{#1}
  \renewcommand{\currentSort}{#2}
  \renewcommand{\currentSize}{#3}
  \renewcommand{\currentAlign}{#4}
  \ifcsname TUserSort@\currentSort\endcsname
    \renewcommand{\currentSortLabel}{\csname TUserSort@\currentSort\endcsname}
  \else
    \renewcommand{\currentSortLabel}{\currentSort}
  \fi
  \begin{scope}[shift={\currentScope}]
  \ifthenelse{\equal{\currentAlign}{l}}{
    \filldraw[process box] (-0.5,-0.5) rectangle (0.5,\currentSize-0.5);
    \node[sort] at (-0.2,\currentSize-0.4) {\currentSortLabel};
   }{\ifthenelse{\equal{\currentAlign}{r}}{
     \filldraw[process box] (-0.5,-0.5) rectangle (0.5,\currentSize-0.5);
     \node[sort] at (0.2,\currentSize-0.4) {\currentSortLabel};
   }{
    \filldraw[process box] (-0.5,-0.5) rectangle (\currentSize-0.5,0.5);
    \ifthenelse{\equal{\currentAlign}{t}}{
      \node[sort,anchor=east] at (-0.3,0.2) {\currentSortLabel};
    }{
      \node[sort] at (-0.6,-0.2) {\currentSortLabel};
    }
   }}
  \setcounter{la}{\currentSize}
  \addtocounter{la}{-1}
  \foreach \i in {0,...,\value{la}} {
    \TProc{\i}
  }
  \end{scope}
}

\newcommand{\TTickProc}[2]{ % pos, label
  \ifthenelse{\equal{\currentAlign}{l}}{
    \draw[tick] (-0.6,#1) -- (-0.4,#1);
    \node[tick label, anchor=east] at (-0.55,#1) {#2};
   }{\ifthenelse{\equal{\currentAlign}{r}}{
    \draw[tick] (0.6,#1) -- (0.4,#1);
    \node[tick label, anchor=west] at (0.55,#1) {#2};
   }{
    \ifthenelse{\equal{\currentAlign}{t}}{
      \draw[tick] (#1,0.6) -- (#1,0.4);
      \node[tick label, anchor=south] at (#1,0.55) {#2};
    }{
      \draw[tick] (#1,-0.6) -- (#1,-0.4);
      \node[tick label, anchor=north] at (#1,-0.55) {#2};
    }
   }}
}
\newcommand{\TSetTick}[3]{
  \expandafter\repcommand\expandafter{\csname TUserTick@#1_#2\endcsname}{#3}
}

\newcommand{\myProc}[3]{
  \ifcsname TUserTick@\currentSort_#1\endcsname
    \TTickProc{#1}{\csname TUserTick@\currentSort_#1\endcsname}
  \else
    \TTickProc{#1}{#1}
  \fi
  \ifthenelse{\equal{\currentAlign}{l}\or\equal{\currentAlign}{r}}{
    \node[#2] (\currentSort_#1) at (0,#1) {#3};
  }{
    \node[#2] (\currentSort_#1) at (#1,0) {#3};
  }
}
\newcommand{\TSetProcStyle}[2]{
  \expandafter\repcommand\expandafter{\csname TUserProcStyle@#1\endcsname}{#2}
}
\newcommand{\TProc}[1]{
  \ifcsname TUserProcStyle@\currentSort_#1\endcsname
    \myProc{#1}{\csname TUserProcStyle@\currentSort_#1\endcsname}{}
  \else
    \myProc{#1}{process}{}
  \fi
}

\newcommand{\repcommand}[2]{
  \providecommand{#1}{#2}
  \renewcommand{#1}{#2}
}
\newcommand{\THit}[5]{
  \path[hit] (#1) edge[#2] (#3#4);
  \expandafter\repcommand\expandafter{\csname TBounce@#3@#5\endcsname}{#4}
}
\newcommand{\TBounce}[4]{
  (#1\csname TBounce@#1@#3\endcsname) edge[#2] (#3#4)
}

%\newcommand{\TState}[1]{
%  \foreach \proc in {#1} {
%    \node[current process] (\proc) at (\proc.center) {};
%  }
%}

\newcommand{\TState}[1]{
  \foreach \proc in {#1} {
        \node[current process] (\proc) at (\proc.center) {};
  };
}
\newcommand{\TCoopHit}[6]{
  \node[#2, apdot] at (#3) {};
  \foreach \proc in {#1} {
    \draw[#2,-] (#3) edge (\proc);
  }
  \path[hit] (#3) edge[#2] (#4#5);
  \expandafter\repcommand\expandafter{\csname TBounce@#4@#6\endcsname}{#5}
}

% ex : \TAction{c_1}{a_1.west}{a_0.north west}{}{right}
% #1 = frappeur
% #2 = cible
% #3 = bond
% #4 = style frappe
% #5 = style bond
\newcommand{\TAction}[5]{
  \THit{#1}{#4}{#2}{}{#3}
  \path[bounce, bend #5=50] \TBounce{#2}{}{#3}{};
}

% ex : \TActionPlur{f_1, c_0}{a_0.west}{a_1.south west}{}{3.5,2.5}{left}
% #1 = frappeur
% #2 = cible
% #3 = bond
% #4 = style frappe
% #5 = coordonnées point central
% #6 = direction bond
\newcommand{\TActionPlur}[6]{
  \TCoopHit{#1}{#4}{#5}{#2}{}{#3}
  \path[bounce, bend #6=50] \TBounce{#2}{}{#3}{};
}

\newdimen\pgfex
\newdimen\pgfem
\usetikzlibrary{arrows,shapes,shadows,scopes}
\usetikzlibrary{positioning}
\usetikzlibrary{matrix}
\usetikzlibrary{decorations.text}
\usetikzlibrary{decorations.pathmorphing}

\usetikzlibrary{arrows,shapes}

\definecolor{lightgray}{rgb}{0.8,0.8,0.8}
\definecolor{lightgrey}{rgb}{0.8,0.8,0.8}

\definecolor{lightred}{rgb}{1,0.8,0.8}
\definecolor{lightgreen}{rgb}{0.7,1,0.7}
\definecolor{darkgreen}{rgb}{0,0.5,0}
\definecolor{darkblue}{rgb}{0,0,0.5}
\definecolor{darkyellow}{rgb}{0.5,0.5,0}
\definecolor{lightyellow}{rgb}{1,1,0.6}
\definecolor{darkcyan}{rgb}{0,0.6,0.6}
\definecolor{lightcyan}{rgb}{0.6,1,1}
\definecolor{darkorange}{rgb}{0.8,0.2,0}
\definecolor{notsodarkred}{rgb}{0.8,0,0}
\definecolor{darkred}{rgb}{0.5,0,0}

\definecolor{notsodarkgreen}{rgb}{0,0.7,0}

%\definecolor{coloract}{rgb}{0,1,0}
%\definecolor{colorinh}{rgb}{1,0,0}
\colorlet{coloract}{darkgreen}
\colorlet{colorinh}{red}
\colorlet{coloractgray}{lightgreen}
\colorlet{colorinhgray}{lightred}
\colorlet{colorinf}{darkgray}
\colorlet{coloractgray}{lightgreen}
\colorlet{colorinhgray}{lightred}

\colorlet{colorgray}{lightgray}
\colorlet{colorhl}{blue}


\tikzstyle{boxed ph}=[]
\tikzstyle{sort}=[fill=lightgray, rounded corners, draw=black]
\tikzstyle{process}=[circle,draw,minimum size=15pt,fill=white,font=\footnotesize,inner sep=1pt]
%\tikzstyle{black process}=[process, draw=blue, fill=red,text=black,font=\bfseries]
\tikzstyle{gray process}=[process, draw=black, fill=lightgray]
\tikzstyle{highlighted process}=[current process, fill=gray]
\tikzstyle{process box}=[fill=none,draw=black,rounded corners]
%\tikzstyle{current process}=[process, draw=black, fill=lightgray]
\tikzstyle{current process}=[process,fill=blue]
\tikzstyle{hl process}=[process,fill=blue!30]
\tikzstyle{tick label}=[font=\footnotesize]
\tikzstyle{tick}=[densely dotted] %-
\tikzstyle{hit}=[->,>=angle 45]
\tikzstyle{selfhit}=[min distance=50pt,curve to]
\tikzstyle{bounce}=[densely dotted,>=stealth',->]
\tikzstyle{ulhit}=[draw=lightgray,fill=lightgray]
\tikzstyle{pulhit}=[fill=lightgray]
\tikzstyle{bulhit}=[draw=lightgray]
\tikzstyle{hl}=[very thick,colorhl]
\tikzstyle{hlb}=[very thick]
\tikzstyle{hlhit}=[hl]
%\tikzstyle{hl2}=[hl]
%\tikzstyle{nohl}=[font=\normalfont,thin]

\tikzstyle{update}=[draw,->,dashed,shorten >=.7cm,shorten <=.7cm]

\tikzstyle{unprio}=[draw,thin]%[double]
%\tikzstyle{prioW}=[draw,thick,-stealth]%[double]
\tikzstyle{prio}=[draw,-stealth,double]

\tikzstyle{hitless graph}=[every edge/.style={draw=red,-}]

\tikzstyle{aS}=[every edge/.style={draw,->,>=stealth}]
\tikzstyle{Asol}=[draw,circle,minimum size=5pt,inner sep=0,node distance=1cm]
\tikzstyle{Aproc}=[draw,node distance=1.2cm]
\tikzstyle{Aobj}=[node distance=1.5cm]
\tikzstyle{Anos}=[font=\Large]

\tikzstyle{AsolPrio}=[Asol,double]
\tikzstyle{AprocPrio}=[Aproc,double]
\tikzstyle{aSPrio}=[aS,double]

\colorlet{colorhlwarn}{notsodarkred}
\colorlet{colorhlwarnbg}{lightred}
\tikzstyle{Ahl}=[very thick,fill=colorhlwarnbg,draw=colorhlwarn,text=colorhlwarn]
\tikzstyle{Ahledge}=[very thick,double=colorhlwarnbg,draw=colorhlwarn,color=colorhlwarn]


\tikzstyle{apdot}=[andot] %[circle, fill=black, draw=black, inner sep=1]
\tikzstyle{apdotsimple}=[] %[circle, fill=black, draw=black, inner sep=1]


%\definecolor{darkred}{rgb}{0.5,0,0}



\tikzstyle{adn}=[every node/.style={circle,draw=black,outer sep=2pt,minimum
                size=15pt,text=black,inner sep=0}, node distance=1.5cm, ->]
\tikzstyle{inh}=[>=|,-|,draw=colorinh,thick, text=black,label]
\tikzstyle{act}=[->,>=triangle 60,draw=coloract,thick,color=coloract]
%\tikzstyle{inhgray}=[>=|,-|,draw=colorinhgray,thick, text=black,label]
%\tikzstyle{actgray}=[->,>=triangle 60,draw=coloractgray,thick,color=coloractgray]
%\tikzstyle{inf}=[->,draw=colorinf,thick,color=colorinf]
%\tikzstyle{elabel}=[fill=none, above=-1pt, sloped,text=black, minimum size=10pt, outer sep=0, font=\scriptsize,draw=none]
\tikzstyle{elabel}=[fill=none,text=black,%sloped,
minimum size=10pt, outer sep=0, font=\scriptsize, draw=none,inner sep=2pt]
%\tikzstyle{elabel}=[]


%\tikzstyle{plot}=[every path/.style={-}]
%\tikzstyle{axe}=[gray,->,>=stealth']
%\tikzstyle{ticks}=[font=\scriptsize,every node/.style={gray}]
%\tikzstyle{mean}=[thick]
%\tikzstyle{interval}=[line width=5pt,red,draw opacity=0.7]
%\definecolor{lightred}{rgb}{1,0.3,0.3}

%\tikzstyle{hl}=[yellow]
%\tikzstyle{hl2}=[orange]

%\tikzstyle{every matrix}=[ampersand replacement=\&]
%\tikzstyle{shorthandoff}=[]
%\tikzstyle{shorthandon}=[]

\tikzstyle{objective}=[process,very thick,fill=yellow!50]

% procedure, abstractions and dependencies
\newcommand{\abstr}[1]{#1^\wedge}%\text{\textasciicircum}}
\def\BS{\mathbf{BS}}
\def\aBS{\abstr{\BS}}
\def\abeta{\abstr{\beta}}
\def\aZ{\abstr{\zeta}}
\def\aY{\abstr{\xi}}

\def\beforeproc{\vartriangleleft}

\def\powerset{\wp}

\def\Sce{\mathbf{Sce}}
\def\OS{\mathbf{OS}}
\def\Obj{\mathbf{Obj}}
%\def\Proc{\mathbf{Proc}}
%\def\Sol{\mathbf{Sol}}
\newcommand{\Sol}{\mathbf{Sol}}

\usepackage{galois}
\newcommand{\theOSabstr}{toOS}
\newcommand{\OSabstr}[1]{\theOSabstr(#1)}
\newcommand{\theOSconcr}{toSce}
\newcommand{\OSconcr}[1]{\theOSconcr(#1)}

% \def\gO{\mathbb{O}}
% \def\gS{\mathbb{S}}
\def\aS{\mathcal{A}}
\def\Req{\mathrm{Req}}
%\def\Sol{\mathrm{Sol}}
\def\Cont{\mathrm{Cont}}
\def\cBS{\BS_\ctx}
\def\caBS{\aBS_\ctx}
\def\caS{\aS_\ctx}
\def\cSol{\Sol_\ctx}
\def\cReq{\Req_\ctx}
\def\cCont{\Cont_\ctx}

\def\any{\star}

% \def\gProc{\mathrm{maxPROC}}
\def\mCtx{\mathrm{maxCtx}}

\def\procs{\f{procs}}
\def\objs{\f{objs}}
\def\sat#1{\lceil #1\rceil}

\def\gCont{\f{maxCont}}
\def\lCont{\f{minCont}}
\def\lProc{\f{minProc}}
\def\gProc{\f{maxProc}}

\def\join{\oplus}
\def\concat{\!::\!}
\def\emptyseq{\varepsilon}
\def\ltw{\preccurlyeq_{\OS}}
\def\indexes#1{\mathbb{I}^{#1}}
%\def\indexes#1{\{1..|#1|\}}
\def\supp{\f{support}}
\def\w{\omega}
\def\W{\Omega}
\def\ctx{\varsigma}
\def\Ctx{\mathbf{Ctx}}
\def\mconcr{\gamma}
\def\concr{\mconcr_\ctx}
\def\obj#1#2{{#1\!\Rsh^*\!\!#2}}
\def\objp#1#2#3{\obj{{#1}_{#2}}{{#1}_{#3}}}
\def\A{\mathcal{A}}
\def\cwA{\A_\ctx^\w}
\def\cwReq{\Req_\ctx^\w}
\def\cwSol{\Sol_\ctx^\w}
\def\cwCont{\Cont_\ctx^\w}
\def\gCtx{\f{maxCtx}}
\def\endCtx{\f{endCtx}}
\def\ceil{\f{end}}

%\def\lfp{\mathrm{lfp}\;}
%\def\mlfp#1{\mathrm{lfp}\{#1\}\;}
\newcommand{\lfp}[3]{\mathbf{lfp}\{#1\}\left(#2\mapsto#3\right)}
\def\maxobjs{{\f{maxobjs}}}
\def\maxprocs{{\f{maxprocs}_\ctx}}
\def\objends{{\f{ends}}}

\def\ra{\rho}
\def\rb{\rho^\wedge}
\def\rc{\widetilde{\rho}}
\def\interleave{\f{interleave}}

\def\join{\concat}

\def\procs{\mathsf{procs}}
%\def\allprocs{\mathsf{allProcs}}
\def\allprocs{\procs}
%\def\pfp{\mathsf{pfp}}
\def\pfp{\mathsf{lst}}
\def\pfpprocs{\mathsf{pfpProcs}}
\def\bounceprocs{\mathsf{bounceProcs}}
\def\newprocs{\mathsf{newProcs}}

\def\aB{\mathcal{B}}
\def\sat#1{\lceil #1\rceil}
\def\cwB{\sat{\aB_\ctx^\w}}
\def\mycwB#1#2{\sat{\aB_{#1}^{#2}}}
\def\Bsol{\sat{\Sol^\w_\ctx}}
\def\Breq{\sat{\Req^\w_\ctx}}
\def\Bcont{\sat{\Cont^\w_\ctx}}

\def\myB{\aB^\w_\ctx}
\def\mysol{\overline{\Sol^\w_\ctx}}
\def\myreq{\overline{\Req^\w_\ctx}}
\def\mycont{\overline{\Cont^\w_\ctx}}

\begin{comment}
\def\PrioCont{\textcolor{red}{\mathrm{PrioCont}}}
\def\mypriocont{\overline{\PrioCont^\w_\ctx}}
\def\cwPrioCont{\PrioCont_\ctx^\w}
\def\Bpriocont{\sat{\PrioCont^\w_\ctx}}
\def\Sat{\PrioCont}
\def\mysat{\overline{\Sat^\w_\ctx}}
\def\cwSat{\Sat_\ctx^\w}
\def\Bsat{\sat{\Sat^\w_\ctx}}

\def\ReqSolPrio{\textcolor{blue}{\mathrm{ReqSolPrio}}}
\def\RSP{\ReqSolPrio}
\def\myrsp{\overline{\RSP^\w_\ctx}}
\def\cwRSP{\RSP_\ctx^\w}
\def\Brsp{\sat{\RSP^\w_\ctx}}
\end{comment}

\newcommand{\csState}{\mathsf{procState}}

\newcommand{\V}{V}
\newcommand{\E}{E}
\newcommand{\cwV}{\V_\ctx^\w}
\newcommand{\cwE}{\E_\ctx^\w}
%\newcommand{\VProc}{\textcolor{red}{\V_\PHproc}}
%\newcommand{\VObj}{\textcolor{red}{\V_\Obj}}
%\newcommand{\VSol}{\V_{Sol}}
%\newcommand{\VSol}{\textcolor{red}{\V_{\Sol}}}
\newcommand{\VProc}{\V \cap \PHproc}
\newcommand{\VObj}{\V \cap \Obj}
\newcommand{\VSol}{\V \cap \Sol}

\def\Bv{\sat{\cwV}}
\def\Be{\sat{\cwE}}
\def\BvProc{\textcolor{red}{\sat{\cwV}^\PHproc}}
\def\BvObj{\textcolor{red}{\sat{\cwV}^\Obj}}
%\def\BvSol{\sat{\cwV}^{Sol}}
\def\BvSol{\textcolor{red}{\sat{\cwV}^{\Sol}}}

\newcommand{\Bee}[2]{\Be^{#1}_{#2}}

%\def\mlfp#1{\f{pppf}\{#1\}}

\def\PHobjp#1#2#3{\PHobj{{#1}_{#2}}{{#1}_{#3}}}
\def\Obj{\mathbf{Obj}}
\def\powerset{\wp}
\def\gCont{\f{maxCont}}

\def\muconcr{\ell}
\def\uconcr{\muconcr_\ctx}

\begin{comment}
\newcommand{\abstr}[1]{#1^\wedge}%\text{\textasciicircum}}
\def\priomax{\mathsf{prio}_{max}}
\def\procs{\mathsf{procs}}
\def\allprocs{\mathsf{allProcs}}
\def\pfp{\mathsf{pfp}}
\def\pfpprocs{\mathsf{pfpProcs}}
%
\def\ctx{\varsigma}
\def\w{\omega}
%\def\aBS{\abstr{\BS}}
%
\def\Req{\mathrm{Req}}
\def\Sol{\mathrm{Sol}}
\def\Cont{\mathrm{Cont}}
\def\A{\mathcal{A}}
\def\cwA{\A_\ctx^\w}
\def\cwReq{\Req_\ctx^\w}
\def\cwSol{\Sol_\ctx^\w}
\def\cwCont{\Cont_\ctx^\w}
%
%
%
\end{comment}

%%% Exemple pour la définition du Process Hitting %%%
\def \exphdef {
\path[use as bounding box] (-0.5,-0.25) rectangle (6.5,4.75);

\TSort{(0,3)}{a}{2}{l}
\TSort{(0,0)}{b}{2}{l}
\TSort{(6,1)}{z}{3}{r}

\THit{a_1}{}{z_1}{.west}{z_2}
\THit{b_1}{}{z_0}{.west}{z_1}
\THit{a_0}{out=250,in=200,selfhit}{a_0}{.west}{a_1}

\path[bounce,bend left]
\TBounce{z_0}{}{z_1}{.south}
\TBounce{z_1}{}{z_2}{.south}
\TBounce{a_0}{}{a_1}{.south}
;
}



%%% Exemple pour la coopération %%%
\def \exphcoop {
\path[use as bounding box] (-0.5,-0.5) rectangle (6.5,4.5);

% Actions de màj grisées
\only<7-8,10-11>{
\THit{a_1}{ulhit,color=lightgray}{ab_0}{.west}{ab_2}
\THit{a_1}{ulhit,color=lightgray}{ab_1}{.160}{ab_3}
\path[bounce,bend left,pulhit] \TBounce{ab_0}{bulhit}{ab_2}{.240} \TBounce{ab_1}{bulhit}{ab_3}{.south west} ;
}

\only<8-8,10-11>{
\THit{a_0}{ulhit}{ab_2}{.160}{ab_0}
\THit{a_0}{ulhit}{ab_3}{.150}{ab_1}
\path[bounce,bend right,pulhit] \TBounce{ab_2}{bulhit}{ab_0}{.north west} \TBounce{ab_3}{bulhit}{ab_1}{.120} ;
}

\only<10-11>{
\THit{b_0}{ulhit}{ab_3}{.190}{ab_2}
\THit{b_0}{ulhit}{ab_1}{.200}{ab_0}
\THit{b_1}{ulhit}{ab_0}{.210}{ab_1}
\THit{b_1}{ulhit}{ab_2}{.200}{ab_3}
\path[bounce,bend right,pulhit] \TBounce{ab_1}{bulhit}{ab_0}{.north} \TBounce{ab_3}{bulhit}{ab_2}{.120} ;
\path[bounce,bend left,pulhit] \TBounce{ab_0}{bulhit}{ab_1}{.240} \TBounce{ab_2}{bulhit}{ab_3}{.south} ;
}

% Sortes
%\TSort{(0,3)}{a}{2}{l}
%\TSort{(0,0)}{b}{2}{l}
%\TSort{(6,1)}{z}{3}{r}
\TSort{(0,3)}{a}{2}{l}
\TSort{(0,0)}{b}{2}{l}
\TSort{(7,1.5)}{z}{2}{r}

% Deux actions disjointes en exemple
\only<2-4>{
\THit{a_1}{}{z_0}{.north west}{z_1}
\path[bounce,bend left] \TBounce{z_0}{}{z_1}{.south} ;

\only<3>{
\THit{a_1}{hl}{z_0}{.north west}{z_1}
\path[bounce,bend left,hl] \TBounce{z_0}{}{z_1}{.south} ;
}

\THit{b_1}{}{z_0}{.west}{z_1}
\path[bounce,bend left=55] \TBounce{z_0}{}{z_1}{.south west} ;
}

% Processus d'exemple
\TState{3}{a_1,b_0,z_0}
\TState{4}{a_1,b_0,z_1}

% Sorte coopérative et arcs
\only<5->{
\TSetTick{ab}{0}{00}
\TSetTick{ab}{1}{01}
\TSetTick{ab}{2}{10}
\TSetTick{ab}{3}{11}
%\TSort{(3,0.5)}{ab}{4}{l}
\TSort{(4,0.5)}{ab}{4}{r}
}



% Actions de màj normales
\only<9,12->{
\THit{a_1}{}{ab_0}{.west}{ab_2}
\THit{a_1}{}{ab_1}{.160}{ab_3}
\path[bounce,bend left] \TBounce{ab_0}{}{ab_2}{.240} \TBounce{ab_1}{}{ab_3}{.south west} ;
\THit{a_0}{}{ab_2}{.160}{ab_0}
\THit{a_0}{}{ab_3}{.150}{ab_1}
\path[bounce,bend right] \TBounce{ab_2}{}{ab_0}{.north west} \TBounce{ab_3}{}{ab_1}{.120} ;
\THit{b_0}{}{ab_3}{.190}{ab_2}
\THit{b_0}{}{ab_1}{.200}{ab_0}
\THit{b_1}{}{ab_0}{.210}{ab_1}
\THit{b_1}{}{ab_2}{.200}{ab_3}
\path[bounce,bend right] \TBounce{ab_1}{}{ab_0}{.north} \TBounce{ab_3}{}{ab_2}{.120} ;
\path[bounce,bend left] \TBounce{ab_0}{}{ab_1}{.240} \TBounce{ab_2}{}{ab_3}{.south} ;
}

% Actions de màj en gras de la sc
\only<6>{
\THit{a_1}{thick}{ab_0}{.west}{ab_2}
\THit{a_1}{thick}{ab_1}{.160}{ab_3}
\path[bounce,thick,bend left] \TBounce{ab_0}{thick}{ab_2}{.240} \TBounce{ab_1}{thick}{ab_3}{.south west} ;
}

\only<7>{
\THit{a_0}{thick}{ab_2}{.160}{ab_0}
\THit{a_0}{thick}{ab_3}{.150}{ab_1}
\path[bounce,thick,bend right] \TBounce{ab_2}{thick}{ab_0}{.north west} \TBounce{ab_3}{thick}{ab_1}{.120} ;
}

\only<8>{
\THit{b_0}{thick}{ab_3}{.190}{ab_2}
\THit{b_0}{thick}{ab_1}{.200}{ab_0}
\THit{b_1}{thick}{ab_0}{.210}{ab_1}
\THit{b_1}{thick}{ab_2}{.200}{ab_3}
\path[bounce,thick,bend right] \TBounce{ab_1}{thick}{ab_0}{.north} \TBounce{ab_3}{thick}{ab_2}{.120} ;
\path[bounce,thick,bend left] \TBounce{ab_0}{thick}{ab_1}{.240} \TBounce{ab_2}{thick}{ab_3}{.south} ;
}

% État d'exemple pour màj de la sc
%\TState{8-9}{a_1,b_1}
%\TState{10}{a_1,b_1,ab_0,ab_1,ab_2,ab_3}
%\TState{11}{a_1,b_1,ab_3}
%\only<9-11>{
%\THit{a_1}{}{ab_0}{.west}{ab_2}
%\THit{a_1}{}{ab_1}{.160}{ab_3}
%\THit{b_1}{}{ab_0}{.210}{ab_1}
%\THit{b_1}{}{ab_2}{.200}{ab_3}
%\path[bounce,bend left] \TBounce{ab_0}{}{ab_2}{.240} \TBounce{ab_1}{}{ab_3}{.south west} ;
%\path[bounce,bend left] \TBounce{ab_0}{}{ab_1}{.240} \TBounce{ab_2}{}{ab_3}{.south} ;
%}

% État d'exemple pour action de la sc
\TState{13}{a_1,b_1,z_0,ab_3}
\TState{14-}{a_1,b_1,z_1,ab_3}

% Arc sortant de la sc
\only<13->{
\THit{ab_3}{thick}{z_0}{.west}{z_1}
\path[bounce,bend left,thick] \TBounce{z_0}{thick}{z_1}{.south} ;
}

% Arc sortant de la sc
%\only<15->{
%\THit{ab_2}{}{z_1}{.west}{z_2}
%\path[bounce,bend left] \TBounce{z_1}{}{z_2}{.south} ;
%}

}



%%% Exemple pour la coopération avec et sans priorités %%%
\def\exphcoopprio#1#2{
%\path[use as bounding box] (-0.5,-0.5) rectangle (6.5,4.5);

% Sortes
\TSort{(0,3)}{a}{2}{l}
\TSort{(0,0)}{b}{2}{l}
\TSort{(7,1.5)}{z}{2}{r}
% Sorte coopérative
\TSetTick{ab}{0}{00}
\TSetTick{ab}{1}{01}
\TSetTick{ab}{2}{10}
\TSetTick{ab}{3}{11}
\TSort{(4,0.5)}{ab}{4}{r}

\ifthenelse{\equal{#2}{abstr}}{
% Abstraction de la màj de la sc
\path[#1,update] (0,3.5) -- (4,2.7) ;
\path[#1,update] (0,0.5) -- (4,1.3) ;
}{
% Actions de màj de la sc
\THit{a_1}{#1}{ab_0}{.west}{ab_2}
\THit{a_1}{#1}{ab_1}{.160}{ab_3}
\path[bounce,bend left] \TBounce{ab_0}{}{ab_2}{.240} \TBounce{ab_1}{}{ab_3}{.south west} ;
\THit{a_0}{#1}{ab_2}{.160}{ab_0}
\THit{a_0}{#1}{ab_3}{.150}{ab_1}
\path[bounce,bend right] \TBounce{ab_2}{}{ab_0}{.north west} \TBounce{ab_3}{}{ab_1}{.120} ;
\THit{b_0}{#1}{ab_3}{.190}{ab_2}
\THit{b_0}{#1}{ab_1}{.200}{ab_0}
\THit{b_1}{#1}{ab_0}{.210}{ab_1}
\THit{b_1}{#1}{ab_2}{.200}{ab_3}
\path[bounce,bend right] \TBounce{ab_1}{}{ab_0}{.north} \TBounce{ab_3}{}{ab_2}{.120} ;
\path[bounce,bend left] \TBounce{ab_0}{}{ab_1}{.240} \TBounce{ab_2}{}{ab_3}{.south} ;
}

%Actions entre a et b
\THit{a_1}{selfhit}{a_1}{.west}{a_0}
\THit{b_1}{selfhit}{b_1}{.west}{b_0}
\THit{a_0}{bend right=50}{b_0}{.west}{b_1}
\THit{b_0.south west}{bend left=90}{a_0}{.west}{a_1}
\path[bounce,bend right] \TBounce{a_1}{}{a_0}{.north west} \TBounce{b_1}{}{b_0}{.north west} ;
\path[bounce,bend left] \TBounce{a_0}{}{a_1}{.south west} \TBounce{b_0}{}{b_1}{.south west} ;

% Arc sortant de la sc
\THit{ab_3}{}{z_0}{.west}{z_1}
\path[bounce,bend left] \TBounce{z_0}{}{z_1}{.south} ;
}



%%% Structure abstraite pour l'atteignabilité avec priorités %%%
\def \priostatic {
\node[Aproc] (z1) {$z_1$};
\node[Aobj,right of=z1] (z01) {$\PHobj{z_0}{z_1}$};
\node[Asol,right of=z01] (z01s) {};

\node[Aproc,right of=z01s] (ab11) {$ab_{11}$};
\node[Asol,right of=ab11] (ab11s) {};

\node[Aproc,above right of=ab11s] (a1) {$a_1$};
\node[Aobj,above right of=a1] (a11) {$\PHobj{a_1}{a_1}$};
\node[Asol,right of=a11] (a11s) {};
\node[Aobj,right of=a1] (a01) {$\PHobj{a_0}{a_1}$};
\node[Asol,right of=a01] (a01s) {};
\node[Aproc,right of=a01s] (b0) {$b_0$};
\node[Aobj,right of=b0] (b00) {$\PHobj{b_0}{b_0}$};
\node[Asol,right of=b00] (b00s) {};
\node[Aobj,above right of=b0] (b10) {$\PHobj{b_1}{b_0}$};
\node[Asol,right of=b10] (b10s) {};

\node[Aproc,below right of=ab11s] (b1) {$b_1$};
\node[Aobj,below right of=b1] (b11) {$\PHobj{b_1}{b_1}$};
\node[Asol,right of=b11] (b11s) {};
\node[Aobj,right of=b1] (b01) {$\PHobj{b_0}{b_1}$};
\node[Asol,right of=b01] (b01s) {};
\node[Aproc,right of=b01s] (a0) {$a_0$};
\node[Aobj,right of=a0] (a00) {$\PHobj{a_0}{a_0}$};
\node[Asol,right of=a00] (a00s) {};
\node[Aobj,below right of=a0] (a10) {$\PHobj{a_1}{a_0}$};
\node[Asol,right of=a10] (a10s) {};

\path
(z1) edge (z01)
(z01) edge (z01s)
(z01s) edge (ab11)
(ab11) edge[aSPrio] (ab11s)
(ab11s) edge (a1) edge (b1)

(a1) edge (a01) edge (a11)
(a01) edge (a01s)
(a01s) edge (b0)
(a11) edge (a11s)
(a0) edge (a10) edge (a00)
(a10) edge (a10s)
(a00) edge (a00s)

(b0) edge (b10) edge (b00)
(b10) edge (b10s)
(b00) edge (b00s)
(b1) edge (b01) edge (b11)
(b01) edge (b01s)
(b01s) edge (a0)
(b11) edge (b11s)
;
}



% Légende pour les GLC
\newcommand{\lvalign}{1 cm}   % Coordinnée X des nœuds
\newcommand{\lvalignl}{2 cm}  % Coordonnée X des légendes
\newcommand{\lhfirst}{3 cm}   % Coordonnée Y du premier élément
\newcommand{\lhsep}{.7 cm}    % Sépérateur Y de chaque élément
\tikzstyle {llegend}=[anchor=west,align=left]
\tikzstyle {legendbox}=[rounded corners,thick,draw=couleurtheme]

\def \glclegend#1#2#3{
  \node[Aproc] at (\lvalign,\lhfirst) (exproc) {#2};
  \node[Aobj] at (\lvalign,\lhfirst-\lhsep) (exobj) {#3};
  \node[Asol] at (\lvalign,\lhfirst-2*\lhsep) (exsol) {};

  \node[llegend] at (\lvalignl,\lhfirst) {Required process};
  \node[llegend] at (\lvalignl,\lhfirst-\lhsep) {Objective};
  \node[llegend] at (\lvalignl,\lhfirst-2*\lhsep) {Solution to an objective};

  \ifthenelse{\equal{#1}{prio}}{%
    \node[Aproc] at (\lvalign-.3cm,\lhfirst-3*\lhsep) (aprio1) {$ab_{11}$};
    \node[Asol] at (\lvalign+.7cm,\lhfirst-3*\lhsep) (aprio2) {};
    \path (aprio1) edge[aSPrio] (aprio2);
    \node[llegend] at (\lvalignl,\lhfirst-3*\lhsep) {\phantom{Mg}\\Solution to a prioritised\\cooperative sort process};
    \path[legendbox] (0,0) rectangle (6,3.6);
  }{%
    \path[legendbox] (0,1) rectangle (6,3.6);
  }
}



%%% Exemple atteignabilité
\def \exatt {
\TSort{(0,0)}{a}{2}{l}
\TSort{(3,0)}{b}{3}{l}
\TSort{(6,0)}{d}{3}{r}
\TSort{(2,-2)}{c}{2}{b}

\THit{a_0}{}{c_0}{.north}{c_1}
\THit{a_1}{}{b_1}{.west}{b_0}
\THit{c_1}{bend left=20pt}{b_0}{.west}{b_1}
\THit{b_1.south west}{->}{a_0}{.east}{a_1}
\THit{b_0}{}{d_0}{.west}{d_1}
\THit{b_1}{}{d_1}{.west}{d_2}
\THit{d_1}{}{b_0}{.north east}{b_2}
\THit{c_1}{bend right=80pt,distance=80pt}{d_1}{.east}{d_0}
\THit{b_2}{distance=120pt,out=30,in=40}{d_0}{.east}{d_2}

\path[bounce,bend left]
\TBounce{d_0}{}{d_1}{.south}
\TBounce{d_1}{}{d_2}{.south}
\TBounce{c_0}{}{c_1}{.west}
\TBounce{b_0}{}{b_1}{.south}
\TBounce{d_1}{}{d_0}{.north}
;
\path[bounce,bend right]
\TBounce{a_0}{}{a_1}{.south}
\TBounce{b_0}{}{b_2}{.south}
\TBounce{b_1}{}{b_0}{.north}
\TBounce{d_0}{bend right=50pt,distance=40pt}{d_2}{.south}
;
}



%%% Figure de présentation de l'analyse d'atteignabilité
\def \figsa {
\begin{tikzpicture}
\path[use as bounding box] (-7,-3.3) rectangle (6.3,3);
\definecolor{r2}{RGB}{238,10,38}

\path<2->[shading=1, inner color=r2, outer color=white] (3.5,-2.8) -- (4.4,3.2) -- (0,3) -- (-4.5,1.4) -- (-2.5,-2.5) -- (0,-3.6) -- (2.8,-2.8);
%\path<2->[shading, inner color=r2, outer color=white, border color=white] (2.8,-2.8) -- (4.5,4.5) -- (0,3.9) -- (-4.5,1.8) -- (-5,-3) -- (0,-3.2) -- (2.8,-2.8);
\draw<2->[thick,fill=white] (2.5,-2.1) -- (3,2.5) -- (-2.7,1.3) -- (-2,-2) -- (2.5,-2.1);
\draw<6->[thick,fill=lightyellow] (2.5,-2.1) -- (3,2.5) -- (-2.7,1.3) -- (-2,-2) -- (2.5,-2.1);

\node<2->[text width=3.5cm, color=red] (s1) at (-5,2) {Over-Approximation};
\path<2->[->,very thick,color=red] (s1.south) edge (-3.5,1.2);
%\node<2->[text width=3cm,color=black] (i1) at (3.7,.2) {$\Rightarrow$};
\node<2->[text width=3cm,color=black] (q) at (4.5,.2) {$\neg Q$};

%\draw<4->[thick, fill=green] (.5,-.8) -- (1,0) -- (.3,1) -- (-1,.5) -- (-.5,-.5) -- (.5,-.8);
\draw<4->[thick, shading=1, top color=darkgreen, bottom color=green] (.5,-.8) -- (1,0) -- (.3,1) -- (-1,.5) -- (-.5,-.5) -- (.5,-.8);
\node<4->[text width=3.5cm,color=darkgreen] (s2) at (5.2,-1.5) {Under-Approximation};
\node<4->[text width=3cm,color=black] (p) at (1.8,.2) {$P$};
%\node<4->[text width=3cm,color=black] (i1) at (2.25,.2) {$\Rightarrow$};

% reaching set
\node[text width=3cm,color=darkcyan] (s) at (1.8,1.7) {Exact solutions};
\node<1->[text width=3cm,color=darkcyan] (s0) at (0,0) {};
\draw[color=darkcyan,very thick] (0,0) ellipse (2 and 1.5);
%\path<1>[draw=white] (2.8,-2.8) -- (4.5,4.5) -- (0,3.9) -- (-4.5,1.8) -- (-5,-3) -- (-2.5,-3.5) -- (0,-3.2) -- (2.8,-2.8);
\node[text width=3cm,color=black] (r) at (2.8,.2) {$R$};

\path<4->[->,very thick,color=darkgreen] (s2) edge (.6,-.4);

\tikzstyle{point}=[circle,draw=red,fill=red,minimum size=5pt,inner sep=0pt]

%\only<5->{
\only<3->{
\node[point] at (-2.4,-2) {};
\node[point] at (-2,2) {};
}
\only<5->{
\node[point] at (0,0) {};
}
\only<7->{
\node[point] at (-.5,-1.1) {};
\node[point] at (2.5,1) {};
}
%}

\end{tikzpicture}
}



%%% Exemples de graphes d'atteignabilité

% Structure abstraite / Sous-approximation / Ok
\def \sauyes {%
\begin{tikzpicture}[aS,node distance=1.1cm]
%\path[use as bounding box] (-0.5,-2.1) rectangle (10.25,2.2);

\node[Aobj] (d02) {$\PHobjectif{d_0}{d_2}$};
\node[Aproc,above of=d02] (d2) {$d_2$};

\node[Asol,right of=d02] (d02s2) {};
\node[Aproc,above right of=d02s2] (b0) {$b_0$};
\node[Aobj,right of=b0] (b10) {$\PHobjectif{b_1}{b_0}$};
\node[Asol,right of=b10] (b10s) {};
\node[Aproc,right of=b10s,node distance=.9cm] (a1) {$a_1$};
\node[Aobj,right of=a1,node distance=1.3cm] (a11) {$\PHobjectif{a_1}{a_1}$};
\node[Asol,right of=a11] (a11s) {};

\node[Aobj,above of=b10,yshift=-0.5cm] (b00)
{$\PHobjectif{b_0}{b_0}$};
\node[Asol,right of=b00] (b00s) {};

\node[Aproc, below of=b0] (b1) {$b_1$};
\node[Aobj,right of=b1] (b11) {$\PHobjectif{b_1}{b_1}$};
\node[Asol,right of=b11] (b11s) {};
\node[Aobj,below of=b11] (b01) {$\PHobjectif{b_0}{b_1}$};
\node[Asol,right of=b01] (b01s) {};
\node[Aproc,right of=b01s,node distance=.9cm] (c1) {$c_1$};
\node[Aobj,right of=c1,node distance=1.3cm] (c11) {$\PHobjectif{c_1}{c_1}$};
\node[Asol,right of=c11] (c11s) {};

\path
(d02) edge (d02s2) (d02s2) edge (b1) edge (b0)
(a11) edge (a11s)
(b10) edge (b10s) (b10s) edge (a1)
(b11) edge (b11s)
(b0) edge (b10) (b1) edge (b11)
(a1) edge (a11)
(d2) edge (d02)
;
\path
(b0) edge (b00.west) (b00) edge (b00s)
(b1) edge (b01)
(b01) edge (b01s) (b01s) edge (c1)
(c1) edge (c11) (c11) edge (c11s)
;
%\node<\tu>[right of=a11s] {\textbf{\Large\color{darkgreen}Yes}};
\end{tikzpicture}%
}

% Structure abstraite / Sous-approximation / Inconclusif
\def \sauinconc {%
\begin{tikzpicture}[aS,node distance=1.1cm]
%\path[use as bounding box] (-0.5,-2.1) rectangle (10.25,2.2);

\node[Aobj] (d02) {$\PHobjectif{d_0}{d_2}$};
\node[Aproc,above of=d02] (d2) {$d_2$};

\node[Asol,right of=d02] (d02s2) {};
\node[Aproc,above right of=d02s2] (b0) {$b_0$};
\node[Aobj,right of=b0] (b10) {$\PHobjectif{b_1}{b_0}$};
\node[Asol,right of=b10] (b10s) {};
\node[Aproc,right of=b10s] (a1) {$a_1$};
\node[Aobj,right of=a1] (a01) {$\PHobjectif{a_0}{a_1}$};
\node[Asol,right of=a01] (a01s) {};

\node[Aproc, below of=b0] (b1) {$b_1$};
\node[Aobj,right of=b1] (b11) {$\PHobjectif{b_1}{b_1}$};
\node[Asol,right of=b11] (b11s) {};
\node[Aobj,below of=b11] (b01) {$\PHobjectif{b_0}{b_1}$};
\node[Asol,right of=b01] (b01s) {};
\node[Aproc,right of=b01s] (c1) {$c_1$};
\node[Aobj,right of=c1] (c01) {$\PHobjectif{c_0}{c_1}$};
\node[Asol,right of=c01] (c01s) {};
\node[Aproc,right of=c01s] (a0) {$a_0$};
\node[Aobj,right of=a0] (a00) {$\PHobjectif{a_0}{a_0}$};
\node[Asol,right of=a00] (a00s) {};

\node[Aobj,above of=b10] (b00) {$\obj{b_0}{b_0}$};
\node[Asol,right of=b00] (b00s) {};
\node[Aobj,above of=a01] (a11) {$\obj{a_1}{a_1}$};
\node[Asol,right of=a11] (a11s) {};
\node[Aobj,above of=c01] (c11) {$\obj{c_1}{c_1}$};
\node[Asol,right of=c11] (c11s) {};
\node[Aobj,above of=a00] (a10) {$\PHobjectif{a_1}{a_0}$};
\node at (a10.east) {\Large\color{red}\textbf{$\bot$}};

\path
  (b10) edge[loop,min distance=5mm] (b10)
 ;
\path
(d02) edge (d02s2) (d02s2) edge (b1) edge (b0)
(a01) edge (a01s) (a01s.south) edge (b1.north east)
(b10) edge (b10s) (b10s) edge (a1)
(b11) edge (b11s)
(a1) edge (a01)
(b0) edge (b10) (b1) edge (b11)
(d2) edge (d02)
;
\path
(b00) edge (b00s)
(b0) edge (b00)
 (b1) edge (b01)
 (b01) edge (b01s) (b01s) edge (c1)
 (c1) edge (c01)
 (c01) edge (c01s) (c01s) edge (a0)
 (a0) edge (a00) (a00) edge (a00s)
;
\path
 (c1) edge (c11) (c11) edge (c11s)
(a0) edge (a10)
(a1) edge (a11)
(a11) edge (a11s)
;

%\node[right of=a01s] {\textbf{\Large\color{darkyellow}Inconc}};

\end{tikzpicture}%
}

% Structure abstraite / Sur-approximation / Non
\def \saono {%
\begin{tikzpicture}[aS,node distance=1.1cm]
%\path[use as bounding box] (-0.5,-2.1) rectangle (10.25,1.15);

\node[Aobj] (d12) {$\PHobjectif{d_1}{d_2}$};
\node[Aproc, above of=d12, node distance=1.5cm] (d2) {$d_2$};
\node[Asol,above right of=d12] (d12s1) {};
\node[Aproc, right of=d12s1] (b2) {$b_2$};
\node[Aobj,right of=b2] (b02) {$\PHobjectif{b_0}{b_2}$};
\node[Asol,right of=b02] (b02s) {};
\node[Aproc,right of=b02s] (d1) {$d_1$};
\node[Aobj,right of=d1] (d11) {$\PHobjectif{d_1}{d_1}$};
\node[Asol,right of=d11] (d11s) {};

\node[Asol,below right of=d12] (d12s2) {};
\node[Aproc, right of=d12s2] (b1) {$b_1$};
\node[Aobj,right of=b1] (b01) {$\PHobjectif{b_0}{b_1}$};
\node[Asol,right of=b01] (b01s) {};
\node[Aproc,right of=b01s] (c1) {$c_1$};
\node[Aobj,right of=c1] (c01) {$\PHobjectif{c_0}{c_1}$};
\node[Asol,right of=c01] (c01s) {};
\node[Aproc,right of=c01s] (a0) {$a_0$};
\node[Aobj,right of=a0] (a10) {$\PHobjectif{a_1}{a_0}$};
\node at (a10.east) {\Large\color{red}\textbf{$\bot$}};

\path
(d2) edge (d12)
(d12) edge (d12s1) edge (d12s2) (d12s1) edge (b2) edge (c1) (d12s2) edge (b1)
(b01) edge (b01s) (b01s) edge (c1)
(b02) edge (b02s) (b02s) edge (d1)
(c01) edge (c01s) (c01s) edge (a0)
(d11) edge (d11s)
(a0) edge (a10)
(b1) edge (b01)
(b2) edge (b02)
(c1) edge (c01)
(d1) edge (d11)
;
%\only<\value{anim1}>{ \node[above right of=c01s] {\textbf{\Large\color{red}No}};}
\end{tikzpicture}%
}

% Structure abstraite / Sur-approximation / Inconclusif
\def \saoinconc {%
\begin{tikzpicture}[aS,node distance=1.1cm]
%\path[use as bounding box] (-0.5,-2.1) rectangle (10.25,1.15);

\node[Aobj] (d02) {$\PHobjectif{d_0}{d_2}$};
\node[Aproc, left of=d02, node distance=1.5cm] (d2) {$d_2$};
\node[Asol,above right of=d02] (d02s1) {};

\node[Aproc, right of=d02s1] (b2) {$b_2$};
\node[Aobj,right of=b2] (b12) {$\PHobjectif{b_1}{b_2}$};
\node[Asol,right of=b12] (b12s) {};
\node[Aproc,right of=b12s] (d1) {$d_1$};
\node[Aobj,right of=d1] (d01) {$\PHobjectif{d_0}{d_1}$};
\node[Asol,right of=d01] (d01s) {};

\node[Asol,below right of=d02] (d02s2) {};
\node[Aproc, right of=d02s2] (b0) {$b_0$};
%\node<\tokp>[orange, thick, Aproc, right of=d02s2] (b0) {$b_0$};
\node[Aobj,right of=b0] (b10) {$\PHobjectif{b_1}{b_0}$};
\node[Asol,right of=b10] (b10s) {};
\node[Aproc,right of=b10s] (a1) {$a_1$};
%\node<\tokp>[orange, thick, Aproc,right of=b10s] (a1) {$a_1$};
\node[Aobj,right of=a1] (a11) {$\PHobjectif{a_1}{a_1}$};
\node[Asol,right of=a11] (a11s) {};

\node[Aproc, below of=b0] (b1) {$b_1$};
\node[Aobj,right of=b1] (b11) {$\PHobjectif{b_1}{b_1}$};
\node[Asol,right of=b11] (b11s) {};

\path
(d2) edge (d02)
(d02) edge (d02s1) edge (d02s2) (d02s1) edge (b2) (d02s2) edge (b1) edge (b0)
(a11) edge (a11s)
(b10) edge (b10s) (b10s) edge (a1)
(b11) edge (b11s)
(b12) edge (b12s) (b12s) edge (d1) edge (a1)
(d01) edge (d01s) (d01s.south) edge (b0)
(a1) edge (a11)
(b0) edge (b10) (b1) edge (b11) (b2) edge (b12)
(d1) edge (d01)
;
%\node[below right of=d01s] {\textbf{\Large\color{yellow}Inconc}};
\end{tikzpicture}%
}



% Structure abstraite / Sur-approximation / Inconclusif / Avec processus-clefs
\def \saoinconckp {%
\begin{tikzpicture}[aS,node distance=1.1cm]
%\path[use as bounding box] (-0.5,-2.1) rectangle (10.25,1.15);

\node[Aobj] (d02) {$\PHobjectif{d_0}{d_2}$};
\node[Aproc, left of=d02, node distance=1.5cm] (d2) {$d_2$};
\node[Asol,above right of=d02] (d02s1) {};

\node[Aproc, right of=d02s1] (b2) {$b_2$};
\node[Aobj,right of=b2] (b12) {$\PHobjectif{b_1}{b_2}$};
\node[Asol,right of=b12] (b12s) {};
\node[Aproc,right of=b12s] (d1) {$d_1$};
\node[Aobj,right of=d1] (d01) {$\PHobjectif{d_0}{d_1}$};
\node[Asol,right of=d01] (d01s) {};

\node[Asol,below right of=d02] (d02s2) {};
\node<-2,4->[Aproc, right of=d02s2] (b0) {$b_0$};
\node<3>[orange, thick, Aproc, right of=d02s2] (b0) {$b_0$};
\node[Aobj,right of=b0] (b10) {$\PHobjectif{b_1}{b_0}$};
\node[Asol,right of=b10] (b10s) {};
\node<-2,4->[Aproc,right of=b10s] (a1) {$a_1$};
\node<3>[orange, thick, Aproc,right of=b10s] (a1) {$a_1$};
\node[Aobj,right of=a1] (a11) {$\PHobjectif{a_1}{a_1}$};
\node[Asol,right of=a11] (a11s) {};

\node[Aproc, below of=b0] (b1) {$b_1$};
\node[Aobj,right of=b1] (b11) {$\PHobjectif{b_1}{b_1}$};
\node[Asol,right of=b11] (b11s) {};

\path
(d2) edge (d02)
(d02) edge (d02s1) edge (d02s2) (d02s1) edge (b2) (d02s2) edge (b1) edge (b0)
(a11) edge (a11s)
(b10) edge (b10s) (b10s) edge (a1)
(b11) edge (b11s)
(b12) edge (b12s) (b12s) edge (d1) edge (a1)
(d01) edge (d01s) (d01s.south) edge (b0)
(a1) edge (a11)
(b0) edge (b10) (b1) edge (b11) (b2) edge (b12)
(d1) edge (d01)
;

\uncover<3>{
\node[orange, font=\bfseries,below of=a11s] (kp) {Key processes};
\path[orange, thick]
        (kp) edge (a1)
        (kp) edge (b0)
;}
%\node[below right of=d01s] {\textbf{\Large\color{yellow}Inconc}};

\node<4->[font=\Huge,red] at (a1) {X};

\end{tikzpicture}%
}



% Exemple pour l'inférence

\def \exphinfblack #1 {
% Sortes
\TSort{(0,3)}{a}{2}{l}
\TSort{(0,0)}{b}{2}{l}
\TSort{(6,0)}{z}{3}{r}

% Sorte coopérative et arcs
\TSetTick{ab}{0}{00}
\TSetTick{ab}{1}{01}
\TSetTick{ab}{2}{10}
\TSetTick{ab}{3}{11}
\TSort{(3,0)}{ab}{4}{l}

% Actions de màj grisées
\THit{a_1}{#1}{ab_0}{.west}{ab_2}
\THit{a_1}{#1}{ab_1}{.west}{ab_3}
\path[bounce,bend left] \TBounce{ab_0}{}{ab_2}{.south} \TBounce{ab_1}{}{ab_3}{.south};

\THit{a_0}{#1}{ab_2}{.west}{ab_0}
\THit{a_0}{#1}{ab_3}{.west}{ab_1}
\path[bounce,bend right] \TBounce{ab_2}{}{ab_0}{.north} \TBounce{ab_3}{}{ab_1}{.north};

\THit{b_0}{#1}{ab_3}{.west}{ab_2}
\THit{b_0}{#1}{ab_1}{.west}{ab_0}
\THit{b_1}{#1}{ab_0}{.west}{ab_1}
\THit{b_1}{#1}{ab_2}{.west}{ab_3}
\path[bounce,bend right] \TBounce{ab_1}{}{ab_0}{.north} \TBounce{ab_3}{}{ab_2}{.north};
\path[bounce,bend left] \TBounce{ab_0}{}{ab_1}{.south} \TBounce{ab_2}{}{ab_3}{.south};

% Arcs sortant de la sc
\THit{ab_2}{}{z_1}{.north west}{z_2}
\THit{ab_2}{}{z_0}{.west}{z_1}
\path[bounce,bend left] \TBounce{z_1}{}{z_2}{.south} \TBounce{z_0}{}{z_1}{.south};

\THit{ab_3}{}{z_2}{.west}{z_1}
\THit{ab_3}{}{z_0}{.west}{z_1}
\THit{ab_1}{}{z_2}{.west}{z_1}
\THit{ab_1}{}{z_0}{.west}{z_1}
\path[bounce,bend left] \TBounce{z_2}{,bend right}{z_1}{.north};

\THit{ab_0}{}{z_2}{.west}{z_1}
\THit{ab_0}{}{z_1}{.south west}{z_0}
\path[bounce,bend right] \TBounce{z_2}{}{z_1}{.north} \TBounce{z_1}{}{z_0}{.north};

}